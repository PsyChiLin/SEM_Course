\documentclass[jou]{apa6}

\usepackage{amssymb,amsmath}
\usepackage{ifxetex,ifluatex}
\usepackage{fixltx2e} % provides \textsubscript
\ifnum 0\ifxetex 1\fi\ifluatex 1\fi=0 % if pdftex
  \usepackage[T1]{fontenc}
  \usepackage[utf8]{inputenc}
\else % if luatex or xelatex
  \ifxetex
    \usepackage{mathspec}
    \usepackage{xltxtra,xunicode}
  \else
    \usepackage{fontspec}
  \fi
  \defaultfontfeatures{Mapping=tex-text,Scale=MatchLowercase}
  \newcommand{\euro}{€}
\fi
% use upquote if available, for straight quotes in verbatim environments
\IfFileExists{upquote.sty}{\usepackage{upquote}}{}
% use microtype if available
\IfFileExists{microtype.sty}{\usepackage{microtype}}{}

% Table formatting
\usepackage{longtable, booktabs}
\usepackage{lscape}
% \usepackage[counterclockwise]{rotating}   % Landscape page setup for large tables
\usepackage{multirow}		% Table styling
\usepackage{tabularx}		% Control Column width
\usepackage[flushleft]{threeparttable}	% Allows for three part tables with a specified notes section
\usepackage{threeparttablex}            % Lets threeparttable work with longtable

% Create new environments so endfloat can handle them
% \newenvironment{ltable}
%   {\begin{landscape}\begin{center}\begin{threeparttable}}
%   {\end{threeparttable}\end{center}\end{landscape}}

\newenvironment{lltable}
  {\begin{landscape}\begin{center}\begin{ThreePartTable}}
  {\end{ThreePartTable}\end{center}\end{landscape}}

  \usepackage{ifthen} % Only add declarations when endfloat package is loaded
  \ifthenelse{\equal{\string jou}{\string man}}{%
   \DeclareDelayedFloatFlavor{ThreePartTable}{table} % Make endfloat play with longtable
   % \DeclareDelayedFloatFlavor{ltable}{table} % Make endfloat play with lscape
   \DeclareDelayedFloatFlavor{lltable}{table} % Make endfloat play with lscape & longtable
  }{}%



% The following enables adjusting longtable caption width to table width
% Solution found at http://golatex.de/longtable-mit-caption-so-breit-wie-die-tabelle-t15767.html
\makeatletter
\newcommand\LastLTentrywidth{1em}
\newlength\longtablewidth
\setlength{\longtablewidth}{1in}
\newcommand\getlongtablewidth{%
 \begingroup
  \ifcsname LT@\roman{LT@tables}\endcsname
  \global\longtablewidth=0pt
  \renewcommand\LT@entry[2]{\global\advance\longtablewidth by ##2\relax\gdef\LastLTentrywidth{##2}}%
  \@nameuse{LT@\roman{LT@tables}}%
  \fi
\endgroup}


\ifxetex
  \usepackage[setpagesize=false, % page size defined by xetex
              unicode=false, % unicode breaks when used with xetex
              xetex]{hyperref}
\else
  \usepackage[unicode=true]{hyperref}
\fi
\hypersetup{breaklinks=true,
            pdfauthor={},
            pdftitle={A Commentary on Steiger (2001)},
            colorlinks=true,
            citecolor=blue,
            urlcolor=blue,
            linkcolor=black,
            pdfborder={0 0 0}}
\urlstyle{same}  % don't use monospace font for urls

\setlength{\parindent}{0pt}
%\setlength{\parskip}{0pt plus 0pt minus 0pt}

\setlength{\emergencystretch}{3em}  % prevent overfull lines


% Manuscript styling
\captionsetup{font=singlespacing,justification=justified}
\usepackage{csquotes}
\usepackage{upgreek}



\usepackage{tikz} % Variable definition to generate author note

% fix for \tightlist problem in pandoc 1.14
\providecommand{\tightlist}{%
  \setlength{\itemsep}{0pt}\setlength{\parskip}{0pt}}

% Essential manuscript parts
  \title{A Commentary on Steiger (2001)}

  \shorttitle{A Commentary on Steiger (2001)}


  \author{Chi-Lin Yu\textsuperscript{1}}

  % \def\affdep{{""}}%
  % \def\affcity{{""}}%

  \affiliation{
    \vspace{0.5cm}
          \textsuperscript{1} Department of Psychology, National Taiwan University  }

  \authornote{
    The present commentary is a homework of Structural Equation Modeling
    course in Department of Psychology, National Taiwan University. Mentor:
    Li-Jen Weng. Student ID: R05227101 (Chi-Lin Yu). E-mail address:
    \href{mailto:r05227101@ntu.edu.tw}{\nolinkurl{r05227101@ntu.edu.tw}};
    \href{mailto:psychilinyu@gmail.com}{\nolinkurl{psychilinyu@gmail.com}}.
    Draft can be found on \url{https://github.com/PsyChiLin}.
  }


  




\usepackage{amsthm}
\newtheorem{theorem}{Theorem}
\newtheorem{lemma}{Lemma}
\theoremstyle{definition}
\newtheorem{definition}{Definition}
\newtheorem{corollary}{Corollary}
\newtheorem{proposition}{Proposition}
\theoremstyle{definition}
\newtheorem{example}{Example}
\theoremstyle{definition}
\newtheorem{exercise}{Exercise}
\theoremstyle{remark}
\newtheorem*{remark}{Remark}
\newtheorem*{solution}{Solution}
\begin{document}

\maketitle

\setcounter{secnumdepth}{0}



\setlength{\parindent}{4ex}

Structural equation modeling (SEM) is a widely used statistical
procedure in psychological research. Several introductory papers and
books have been written to convey basic understandings of SEM in various
domains. However, several significant practical and theoretical issues
still exists (Steiger, 2001). Steiger (2001) has presented some comments
on four published textbooks, including three introductory textbooks
(Kelloway, 1998; Kline, 1998; Maruyama, 1998) and a edited volume
(Schumacker \& Marcoulides, 1998). Although the review article (Steiger,
2001) was written in 2001 (almost 20 years ago), some of presented
concepts and reminders can still be applied to textbooks or articles
nowadays. Here, the present study reviewed the work done by Steiger
(2001) and provided some own perspectives on four of the topics in the
article. The rest of the present paper is organized as four proposed
viewpoints and ends with a conclusion section.

The first point I would like to discussion is about model specification,
or theoretical formulation. Steiger (2001) provided some important
arguments regarding theoretical formulation in the \enquote{Equivalent
Models and Path Reversibility} section. He noted the existences of
equivalent models and suggested that SEM users should understand this
point while presenting a casual interpretation. Also, he argued that the
model would become uninterpretable if the direct effect between two
constructs could be reversed without influencing model fitting. I
strongly agreed with his perspectives. In the theoretical formulation
stage, researcher often used correlational studies to establish the
linkage between constructs. Examples included several articles in the
reanalysis presentation in our SEM class. For example, in a paper,
\enquote{Social Perception as a Mediator of the Influence of Early
Visual Processing on Functional Status in Schizophrenia}, the proposed
mediation model was constructed by several correlational studies.
However, we understand that the direction of an arrow between constructs
could not be implied by correlational studies only. That is, although
the association between A construct (e.g., social perception) and B
construct (e.g., functional status) have been frequently reported, we
still could not confirm the direction of the linkage between these two
constructs. In addition, empirical research often presented some SEM
models that have equivalent confirmatory factor analysis (CFA)
counterparts, implying that the relationship between constructs could
mainly be captured by using a CFA model without considering direct
effects in a SEM model. Example also included several articles in the
reanalysis presentation (Sergi, Rassovsky, Nuechterlein, \& Green, 2006;
Wright \& Perrone, 2010). Nevertheless, this issue may hinder the
interpretations of the model. Thus, I think that every SEM users should
not only understand the existence of equivalent models, but also notice
them in their own models even before collecting data. Since the
theoretical formulation stage is the most important stage in a SEM
study, researcher should carefully review relevant articles and identify
possible problems in the formulations. It may strongly influence how
strong the conclusions we can draw in a study.

The second issue that should be mentioned and discussed is the sample
size issue. Steiger (2001) provided some brief comments on the
\enquote{Power and Sample Size Analysis} section. He mentioned that
researchers should perform Monte Carlo simulation or some other
techniques to assess sample size. I agree that \enquote{Choice of an
appropriate sample size is critical in any multivariate analysis}
(Steiger, 2001), and I also believe that it is particularly true in the
domain of SEM since SEM is developed based on asymptotic theory. For
example, the most popular estimation methods, maximum likelihood, has
several desire properties, including unbiasedness, consistency,
efficiency and normality, when the assumptions of large sample size
hold. If researchers do not understand the importance of sample size and
start to perform SEM with samll sample size, the desire properties may
not exist and the implications of the analyses may be incorrect.
Although how large the sample size is large enough is still under
debate, I think that all the textbook or introductory papers must
mention the asymptotic theory of SEM and the importance of sample size
in SEM. Furthermore, as mentioned in Steiger (2001), power analysis
could be provided in some advanced sections.

Third, the reporting techniques are interesting topics to be discussed.
Some articles with SEM were somehow controversial, because some
researcher chose to report the results, e.g, fit indices, which favor
their hypotheses. Some might use strategies to avoid criticism. Also,
few articles report enough information for other researchers to
replicate the analyses. Steiger (2001) provided his opinion that every
SEM textbooks should include a section with example of what to do and
what not to do when reporting the results of SEM. I agree with this
argument but provide a extension. I think every journals, which accepts
SEM related papers, should provide their own templates or guidelines of
reporting SEM. Editors should state that papers that do not contain some
particular information will not be accepted (even not be reviewed). On
the one hand, since every researchers have their own reporting styles
that may not be specifically appropriate for the journal, providing a
reporting template could save some time for researchers. I believe that
journals are responsible for doing this. On the other hand, journals can
decide what kind of information they want to see, what kind of
information should be place in supplementary files, and what kind of
information can be neglected. It would allow journals to establish their
own styles and make all the submitted articles as close to the scope of
the journal as possible. If journals can provide such information,
either a template or a guide, it will be extremely helpful for users to
focus on content related information rather than reporting or
formatting.

Last but not least, a comment that Steiger (2001) made on Schumacker and
Marcoulides (1998) is really significant for SEM educations. Steiger
(2001) mentioned that there is no information of a link or site where
data files can be downloaded, and the example in the book lacks some
beneficial details. I think that it is really important for a textbook
or a tutorial paper to present their sample data and details to
replicate the examples. It is called \enquote{minimum reproducible
example}. These arguments are not restricted to SEM analysis but applied
to all kinds of methodology. A good tutorial should allow users to
easily access to the core of a method. If the data in the example is not
available, how can a beginner strat ? Also, if there is no enough detail
about the example, users may not be able to learn from the example. In
my own experience (of course a personal one), any new methodology
without providing an open sample dataset to perform the demonstrations
would not be accepted by journals. I have requested to provide the
sample dataset to support new methods twice (and one is already
rejected\ldots{}), even if I have provided all the source code and lots
of descriptions (Hope that my paper can be accepted after providing a
sample dataset\ldots{}). Thus, I now think that making data/code
available is essential for tutorial purposes. To be more general, in my
opinion, every papers should make the dataset available (not only in SEM
domain or for educational purposes). Several societies have provided
enough space and platform to allow researchers to share their
code/data/ideas/manuscripts. Examples include Github, zenodo
(\url{https://zenodo.org/}), Open Science Framework (OSF,
\url{https://osf.io/}), and so on. I believe that it is a good way to
increase scientific progress through \enquote{replications} and
\enquote{reproducible research}. At least, back to the comments of
Steiger (2001), clear descriptions and reproducible examples with
available dataset could facilitate the educations of SEM.

In summary, the present commentary reviewed four interesting topics
discussed in Steiger (2001) and provided own viewpoints, from
theoretical relevant to tutorial relevant topics. I think most of the
comments in Steiger (2001) can be applied to other multivariate
statistical textbooks, and most of the discussion in the present study
can be applied to recent research. The present study believe that all of
the issues can be addressed through the successful interaction between
researchers, either an expert or a beginner, as long as all of us hope
to benefit the scientific progress.

\hypertarget{references}{%
\section{References}\label{references}}

\setlength{\parindent}{-0.5in}
\setlength{\leftskip}{0.5in}

\hypertarget{refs}{}
\leavevmode\hypertarget{ref-kelloway1998using}{}%
Kelloway, E. K. (1998). \emph{Using lisrel for structural equation
modeling: A researcher's guide}. Sage.

\leavevmode\hypertarget{ref-kline1998principles}{}%
Kline, R. (1998). Principles and practice of structural equation
modeling, 3rd edn guilford press. \emph{New York}.

\leavevmode\hypertarget{ref-maruyama1998basics}{}%
Maruyama, G. (1998). \emph{Basics of structural equation modeling}.
Sage.

\leavevmode\hypertarget{ref-schumacker1998interaction}{}%
Schumacker, R. E., \& Marcoulides, G. A. (1998). Interaction and
nonlinear effects in structural equation modeling.

\leavevmode\hypertarget{ref-sergi2006social}{}%
Sergi, M. J., Rassovsky, Y., Nuechterlein, K. H., \& Green, M. F.
(2006). Social perception as a mediator of the influence of early visual
processing on functional status in schizophrenia. \emph{American Journal
of Psychiatry}, \emph{163}(3), 448--454.

\leavevmode\hypertarget{ref-steiger2001driving}{}%
Steiger, J. H. (2001). Driving fast in reverse. \emph{Journal of the
American Statistical Association}, \emph{96}(453), 331--338.

\leavevmode\hypertarget{ref-wright2010examination}{}%
Wright, S. L., \& Perrone, K. M. (2010). An examination of the role of
attachment and efficacy in life satisfaction. \emph{The Counseling
Psychologist}, \emph{38}(6), 796--823.






\end{document}
