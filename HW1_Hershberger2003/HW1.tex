\documentclass[jou]{apa6}

\usepackage{amssymb,amsmath}
\usepackage{ifxetex,ifluatex}
\usepackage{fixltx2e} % provides \textsubscript
\ifnum 0\ifxetex 1\fi\ifluatex 1\fi=0 % if pdftex
  \usepackage[T1]{fontenc}
  \usepackage[utf8]{inputenc}
\else % if luatex or xelatex
  \ifxetex
    \usepackage{mathspec}
    \usepackage{xltxtra,xunicode}
  \else
    \usepackage{fontspec}
  \fi
  \defaultfontfeatures{Mapping=tex-text,Scale=MatchLowercase}
  \newcommand{\euro}{€}
\fi
% use upquote if available, for straight quotes in verbatim environments
\IfFileExists{upquote.sty}{\usepackage{upquote}}{}
% use microtype if available
\IfFileExists{microtype.sty}{\usepackage{microtype}}{}

% Table formatting
\usepackage{longtable, booktabs}
\usepackage{lscape}
% \usepackage[counterclockwise]{rotating}   % Landscape page setup for large tables
\usepackage{multirow}		% Table styling
\usepackage{tabularx}		% Control Column width
\usepackage[flushleft]{threeparttable}	% Allows for three part tables with a specified notes section
\usepackage{threeparttablex}            % Lets threeparttable work with longtable

% Create new environments so endfloat can handle them
% \newenvironment{ltable}
%   {\begin{landscape}\begin{center}\begin{threeparttable}}
%   {\end{threeparttable}\end{center}\end{landscape}}

\newenvironment{lltable}
  {\begin{landscape}\begin{center}\begin{ThreePartTable}}
  {\end{ThreePartTable}\end{center}\end{landscape}}

  \usepackage{ifthen} % Only add declarations when endfloat package is loaded
  \ifthenelse{\equal{\string jou}{\string man}}{%
   \DeclareDelayedFloatFlavor{ThreePartTable}{table} % Make endfloat play with longtable
   % \DeclareDelayedFloatFlavor{ltable}{table} % Make endfloat play with lscape
   \DeclareDelayedFloatFlavor{lltable}{table} % Make endfloat play with lscape & longtable
  }{}%



% The following enables adjusting longtable caption width to table width
% Solution found at http://golatex.de/longtable-mit-caption-so-breit-wie-die-tabelle-t15767.html
\makeatletter
\newcommand\LastLTentrywidth{1em}
\newlength\longtablewidth
\setlength{\longtablewidth}{1in}
\newcommand\getlongtablewidth{%
 \begingroup
  \ifcsname LT@\roman{LT@tables}\endcsname
  \global\longtablewidth=0pt
  \renewcommand\LT@entry[2]{\global\advance\longtablewidth by ##2\relax\gdef\LastLTentrywidth{##2}}%
  \@nameuse{LT@\roman{LT@tables}}%
  \fi
\endgroup}


\ifxetex
  \usepackage[setpagesize=false, % page size defined by xetex
              unicode=false, % unicode breaks when used with xetex
              xetex]{hyperref}
\else
  \usepackage[unicode=true]{hyperref}
\fi
\hypersetup{breaklinks=true,
            pdfauthor={},
            pdftitle={Moving Beyond Academic Echo Chambers of Structural Equation Modeling: A Commentary on Hershberger (2003)},
            colorlinks=true,
            citecolor=blue,
            urlcolor=blue,
            linkcolor=black,
            pdfborder={0 0 0}}
\urlstyle{same}  % don't use monospace font for urls

\setlength{\parindent}{0pt}
%\setlength{\parskip}{0pt plus 0pt minus 0pt}

\setlength{\emergencystretch}{3em}  % prevent overfull lines


% Manuscript styling
\captionsetup{font=singlespacing,justification=justified}
\usepackage{csquotes}
\usepackage{upgreek}



\usepackage{tikz} % Variable definition to generate author note

% fix for \tightlist problem in pandoc 1.14
\providecommand{\tightlist}{%
  \setlength{\itemsep}{0pt}\setlength{\parskip}{0pt}}

% Essential manuscript parts
  \title{Moving Beyond Academic Echo Chambers of Structural Equation Modeling: A
Commentary on Hershberger (2003)}



  \author{Chi-Lin Yu\textsuperscript{1}}

  % \def\affdep{{""}}%
  % \def\affcity{{""}}%

  \affiliation{
    \vspace{0.5cm}
          \textsuperscript{1} Department of Psychology, National Taiwan University  }

  \authornote{
    The present commentary is a homework (HW1) of Structural Equation
    Modeling course in Department of Psychology, National Taiwan University.
    Mentor: Li-Jen Weng. Student ID: R05227101 (Chi-Lin Yu). E-mail address:
    \href{mailto:r05227101@ntu.edu.tw}{\nolinkurl{r05227101@ntu.edu.tw}};
    \href{mailto:psychilinyu@gmail.com}{\nolinkurl{psychilinyu@gmail.com}}.
  }


  




\usepackage{amsthm}
\newtheorem{theorem}{Theorem}
\newtheorem{lemma}{Lemma}
\theoremstyle{definition}
\newtheorem{definition}{Definition}
\newtheorem{corollary}{Corollary}
\newtheorem{proposition}{Proposition}
\theoremstyle{definition}
\newtheorem{example}{Example}
\theoremstyle{definition}
\newtheorem{exercise}{Exercise}
\theoremstyle{remark}
\newtheorem*{remark}{Remark}
\newtheorem*{solution}{Solution}
\begin{document}

\maketitle

\setcounter{secnumdepth}{0}



Structural equation modeling (SEM) is a widely used statistical
procedure in psychological research. Over the past decades, SEM
applications are published in numerous research both substantive one
adopted SEM to analyze real data and the technical one that developed
new progress of SEM. Nowatoday, interests in SEM is high and continue to
grow (REF). Here, the present study reviewed the work doned by
Hershberger (2003), which reviewed the growth of SEM from 1994 to 2001,
to get a deeper understandings of the SEM progress in the literature. In
addition, several own perspectives about the paper (REF) and SEM
methodlogy itself are provided. In other words, the rest of the present
paper is organized as a quick review for Hershberger (2003), followed by
a series of proposed viewpoints and ends with a conclusion section.

Hershberger (2003) reviewed the growth of SEM from 1994 to 2001. They
used \emph{PsycIFO} database to locate journal articles published during
these seven years, and both substantive and technical research with SEM
techiques were identified. Also, they specifically examine the presence
of SEM papers in American Psychological Association (APA) journals.
Likewise, they examined the development of a SEM-specific journal,
\emph{Structural Equation Modeling: A Muvtltidisciplinary Journal},
during this period. The results showed the that number of SEM articles,
both the number of substantive articles and the number of techinical
articles, increased during this period. The technical articles that
published in \emph{Structural Equation Modeling: A Multidisciplinary
Journal} contributed as much as all other journals combined. Forty-seven
categroies were created to characterize the development of SEM
methodoloy. SEM was also identified to have the most consistently high
level of development relative to other multivariate statistical tools,
such as exploratory factor analysis. Overall, hershberger (2003)
suggested that SEM not only could be considered as the most popular tool
for multivariate method of data analysis, but also had a stable
methodological development with pace of the use in practical research.

\newpage

\hypertarget{references}{%
\section{References}\label{references}}

\setlength{\parindent}{-0.5in}
\setlength{\leftskip}{0.5in}






\end{document}
