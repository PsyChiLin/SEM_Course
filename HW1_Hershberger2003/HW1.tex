\documentclass[jou]{apa6}

\usepackage{amssymb,amsmath}
\usepackage{ifxetex,ifluatex}
\usepackage{fixltx2e} % provides \textsubscript
\ifnum 0\ifxetex 1\fi\ifluatex 1\fi=0 % if pdftex
  \usepackage[T1]{fontenc}
  \usepackage[utf8]{inputenc}
\else % if luatex or xelatex
  \ifxetex
    \usepackage{mathspec}
    \usepackage{xltxtra,xunicode}
  \else
    \usepackage{fontspec}
  \fi
  \defaultfontfeatures{Mapping=tex-text,Scale=MatchLowercase}
  \newcommand{\euro}{€}
\fi
% use upquote if available, for straight quotes in verbatim environments
\IfFileExists{upquote.sty}{\usepackage{upquote}}{}
% use microtype if available
\IfFileExists{microtype.sty}{\usepackage{microtype}}{}

% Table formatting
\usepackage{longtable, booktabs}
\usepackage{lscape}
% \usepackage[counterclockwise]{rotating}   % Landscape page setup for large tables
\usepackage{multirow}		% Table styling
\usepackage{tabularx}		% Control Column width
\usepackage[flushleft]{threeparttable}	% Allows for three part tables with a specified notes section
\usepackage{threeparttablex}            % Lets threeparttable work with longtable

% Create new environments so endfloat can handle them
% \newenvironment{ltable}
%   {\begin{landscape}\begin{center}\begin{threeparttable}}
%   {\end{threeparttable}\end{center}\end{landscape}}

\newenvironment{lltable}
  {\begin{landscape}\begin{center}\begin{ThreePartTable}}
  {\end{ThreePartTable}\end{center}\end{landscape}}

  \usepackage{ifthen} % Only add declarations when endfloat package is loaded
  \ifthenelse{\equal{\string jou}{\string man}}{%
   \DeclareDelayedFloatFlavor{ThreePartTable}{table} % Make endfloat play with longtable
   % \DeclareDelayedFloatFlavor{ltable}{table} % Make endfloat play with lscape
   \DeclareDelayedFloatFlavor{lltable}{table} % Make endfloat play with lscape & longtable
  }{}%



% The following enables adjusting longtable caption width to table width
% Solution found at http://golatex.de/longtable-mit-caption-so-breit-wie-die-tabelle-t15767.html
\makeatletter
\newcommand\LastLTentrywidth{1em}
\newlength\longtablewidth
\setlength{\longtablewidth}{1in}
\newcommand\getlongtablewidth{%
 \begingroup
  \ifcsname LT@\roman{LT@tables}\endcsname
  \global\longtablewidth=0pt
  \renewcommand\LT@entry[2]{\global\advance\longtablewidth by ##2\relax\gdef\LastLTentrywidth{##2}}%
  \@nameuse{LT@\roman{LT@tables}}%
  \fi
\endgroup}


\ifxetex
  \usepackage[setpagesize=false, % page size defined by xetex
              unicode=false, % unicode breaks when used with xetex
              xetex]{hyperref}
\else
  \usepackage[unicode=true]{hyperref}
\fi
\hypersetup{breaklinks=true,
            pdfauthor={},
            pdftitle={Moving Beyond Academic Echo Chambers of Structural Equation Modeling: A Commentary on Hershberger (2003)},
            colorlinks=true,
            citecolor=blue,
            urlcolor=blue,
            linkcolor=black,
            pdfborder={0 0 0}}
\urlstyle{same}  % don't use monospace font for urls

\setlength{\parindent}{0pt}
%\setlength{\parskip}{0pt plus 0pt minus 0pt}

\setlength{\emergencystretch}{3em}  % prevent overfull lines


% Manuscript styling
\captionsetup{font=singlespacing,justification=justified}
\usepackage{csquotes}
\usepackage{upgreek}



\usepackage{tikz} % Variable definition to generate author note

% fix for \tightlist problem in pandoc 1.14
\providecommand{\tightlist}{%
  \setlength{\itemsep}{0pt}\setlength{\parskip}{0pt}}

% Essential manuscript parts
  \title{Moving Beyond Academic Echo Chambers of Structural Equation Modeling: A
Commentary on Hershberger (2003)}



  \author{Chi-Lin Yu\textsuperscript{1}}

  % \def\affdep{{""}}%
  % \def\affcity{{""}}%

  \affiliation{
    \vspace{0.5cm}
          \textsuperscript{1} Department of Psychology, National Taiwan University  }

  \authornote{
    The present commentary is a homework (HW1) of Structural Equation
    Modeling course in Department of Psychology, National Taiwan University.
    Mentor: Li-Jen Weng. Student ID: R05227101 (Chi-Lin Yu). E-mail address:
    \href{mailto:r05227101@ntu.edu.tw}{\nolinkurl{r05227101@ntu.edu.tw}};
    \href{mailto:psychilinyu@gmail.com}{\nolinkurl{psychilinyu@gmail.com}}.
  }


  




\usepackage{amsthm}
\newtheorem{theorem}{Theorem}
\newtheorem{lemma}{Lemma}
\theoremstyle{definition}
\newtheorem{definition}{Definition}
\newtheorem{corollary}{Corollary}
\newtheorem{proposition}{Proposition}
\theoremstyle{definition}
\newtheorem{example}{Example}
\theoremstyle{definition}
\newtheorem{exercise}{Exercise}
\theoremstyle{remark}
\newtheorem*{remark}{Remark}
\newtheorem*{solution}{Solution}
\begin{document}

\maketitle

\setcounter{secnumdepth}{0}



Structural equation modeling (SEM) is a widely used statistical
procedure in psychological research. Over the past decades, SEM
applications are applied to numerous research, including both
substantive articles that adopted SEM to analyze real data and the
technical one that developed new progress of SEM. Nowadays, interests in
SEM is still high and continue to grow (Bollen \& Pearl, 2013). Here,
the present study reviewed the work done by Hershberger (2003), which
examined the growth of SEM from 1994 to 2001, to get a deeper
understanding of the SEM progress in the literature. In addition,
several own perspectives about the paper (Hershberger, 2003) and SEM
itself were provided. The rest of the present paper is organized as a
quick review for Hershberger (2003), followed by a series of proposed
viewpoints and ends with a conclusion section.

\hypertarget{the-review}{%
\section{The Review}\label{the-review}}

Hershberger (2003) investigated the growth of SEM from 1994 to 2001.
They used \emph{PsycIFO} database to locate journal articles published
during these seven years. Both substantive and technical research with
SEM techniques were identified. Also, they specifically examine the
presence of SEM papers in American Psychological Association (APA)
journals. Likewise, the development of a SEM-specific journal,
\emph{Structural Equation Modeling: A Multidisciplinary Journal}, during
this period was studied. The results showed that the number of SEM
articles, both the number of substantive articles and the number of
technical articles, increased during this period. The technical articles
that published in \emph{Structural Equation Modeling: A
Multidisciplinary Journal} contributed as much as all other journals
combined. Forty-seven technical categories were created to characterize
the development of SEM methodology. SEM was also identified to have the
most consistently high level of development relative to other
multivariate statistical tools, such as exploratory factor analysis.
Overall, Hershberger (2003) suggested that SEM not only could be
considered as the most popular tool for multivariate method of data
analysis, but also had a stable methodological development with pace of
the use in practical research.

\hypertarget{the-encouraging-growth}{%
\section{The Encouraging Growth}\label{the-encouraging-growth}}

The most encouraging finding was the stable progress of SEM. The present
study examined its progress from four dimensions.

\setlength{\parindent}{4ex}

First of all, the number of SEM articles increased, which can be
obviously supported from the calculations in Hershberger (2003). During
these seven years, the number of substantive articles increased from 148
to 335, and the number of technical articles increased from 18 to 46.

Second, the number of technical subjects was expanded from 13 in 1994 to
almost 30 in 2001, evidencing the interests in SEM techniques become
more and more diverse. Although some fundamental topics were
consistently discussed in the literature, such as model specification
and goodness of fit, some novel but important topics were emerged then.
For example, the topic of scale invariance were first discussed in 2001
as Hershberger (2003) mentioned, and this issue became really popular in
recent years. For instance, the journal \emph{Chinese Journal of
Psychology} announced a special issue in 2018 to call for papers on the
scale invariance related topics. It suggested the improvements of SEM
became much more extensive.

Third, the softwares for SEM also increased. Before 1974, the first
software package for SEM, \texttt{LISREL} (Joreskog \& Sorbom, 1996),
was developed. It is still one of the most widely used software packages
for SEM today. Some more popular commercial softwares, including
\texttt{EQS} (Bentler, 1995), \texttt{AMOS} (Arbuckle, 2011), and
\texttt{Mplus} (Muthén \& Muthén, 2010), were developed to support SEM
applications. With the advance of SEM methodology, several open source
tools were also designed to make all researchers have the possibility to
use and get better understandings of SEM. One of the most well known
example is \texttt{lavaan} (Rosseel, 2012), a free, open-source, and
high-quality package that has provided researchers a very good
alternative for latent variable modeling. In general, the number of
softwares with a lot of citations also evidenced the development of SEM.

Last but not least, since the emerge of SEM, its use had permeated
fields from psychology, business, education to even neuroimaging domain.
For example, a structural equation modeling approach for fMRI data (Kim,
Zhu, Chang, Bentler, \& Ernst, 2007) were proposed in 2007, allowing the
explorations of effective connectivity maps with event-related fMRI. In
recent years, the development of SEM in neuroimaging domain was still
ongoing (Gates, Molenaar, Hillary, \& Slobounov, 2011). Therefore, the
widely used of SEM in several different domains can strongly support its
popularity and continuous progress.

Overall, from these four dimensions, it was no doubt that there is an
highly active research community that had successfully promoted the
progress of SEM.

\hypertarget{the-rooms-for-improvements}{%
\section{The Rooms for Improvements}\label{the-rooms-for-improvements}}

\noindent Although the growth of SEM have been evidenced (Hershberger,
2003) and SEM as a great tool has provided better characterization of
data, it is equally important to let all SEM users understand its
technical progress and how to correctly use SEM/explain the results of
SEM.

\setlength{\parindent}{4ex}

Regarding the technical progress of SEM, most of these important
articles were published in \emph{Structural Equation Modeling: A
Multidisciplinary Journal}, and others were published in
methodology-specific journals, such as \emph{Psychometrika} and
\emph{Multivariate Behavioral Research}. Though the publications in
those specific journals can provided important contributions to SEM
progress, it is also clear that applied users without strong
methodological background would not be interested in those journals. The
applied researchers are more likely to directly use SEM rather than
understanding its technical details. Thus, the latest SEM progress can
not be easily promoted and adopted in practical research.

In addition, despite the widespread use of SEM, misunderstandings
regarding its purpose and methods persist. In 2013, a chapter entitled
\enquote{Eight Myths About Causality and Structural Equation Models}
(Bollen \& Pearl, 2013) presented eight major misunderstandings of SEM.
Those myths contains key procedures of SEM from the purpose (e.g.,
causal relations), the model, to interpretations. It seems that user and
researchers (even reviewers), who used SEM to establish their theories
or tell their stories, do need a lot more training on SEM in order to
overcome the misinformation.

Perhaps one of the direct solutions to the aforementioned issues is to
publish papers in applied journals (e.g., \emph{ Psychological Science})
rather than only in methodological journals (e.g., \emph{Structural
Equation Modeling: A Multidisciplinary Journal}). Publish a paper in a
more general journal can elicit more discussion and make more people to
understand new issues or new methods. I used an article in fMRI domain
as an example. An article in \emph{Proceedings of the National Academy
of Sciences} described some methodological issues in fRMI domain
(Eklund, Nichols, \& Knutsson, 2016). It raised lot of discussions and
was cited a lot. Most of all, it allowed fMRI researchers to reconsider
their analyses. Obviously, the effect would not blow up if they chose to
publish their article in a methodology-specific journal
\emph{Neuroinformatics}. Of course, the impact factor and equality of
these two example journals are not the same; but, the present commentary
just wanted to point out that a general journal might also be a good
place to spread the methodological ideas. For example, it might get much
more \enquote{reads} from applied researchers in \emph{Psychological
Science} than in \emph{Psychological Methods}.

Some may argue that those general journals do not accept methodological
papers, and applied users still do not want to understand technical
details. The present commentary speculate that the \enquote{commentary}
section in these journals might be a possible choice. That is, the main
methodological details can be presented in methodology-specific
journals, and also send a brief article that includes all the important
parts (i.e., guideline for new method, suggestions..etc) as a commentary
to general journals. Users can first get a quick impression of the new
progress of SEM and realize how to apply those progress to their ongoing
studies, and then users who want to get a better understanding can read
through the full paper with methodological details.

In order to prevent SEM from becoming a widely used but poorly
understood statistical procedure, the present commentary believed that
it is important to improve the communication between methodologists and
users. With successful interaction between methodologists and users, not
only SEM methodology can have stable development with pace of the
widespread use in practical research, the future use of SEM in practical
research can also catch up with the progress of SEM methodology.

\hypertarget{conclusion}{%
\section{Conclusion}\label{conclusion}}

\noindent The present commentary reviewed the findings in Hershberger
(2003) and provided own viewpoints. Four dimensions for the growth of
SEM were discussed. Most of all, the rooms for improvements were also
suggested. The present study sincerely hopes that SEM community can move
beyond academic echo chambers, further expanding the energy and all
novel usages of SEM to all researchers.

\newpage

\hypertarget{references}{%
\section{References}\label{references}}

\setlength{\parindent}{-0.5in}
\setlength{\leftskip}{0.5in}

\hypertarget{refs}{}
\leavevmode\hypertarget{ref-arbuckle2011ibm}{}%
Arbuckle, J. (2011). IBM spss amos 20 user's guide: IBM corporation.

\leavevmode\hypertarget{ref-bentler1995eqs}{}%
Bentler, P. M. (1995). \emph{EQS structural equations program manual}.
Multivariate software.

\leavevmode\hypertarget{ref-bollen2013eight}{}%
Bollen, K. A., \& Pearl, J. (2013). Eight myths about causality and
structural equation models. In \emph{Handbook of causal analysis for
social research} (pp. 301--328). Springer.

\leavevmode\hypertarget{ref-eklund2016cluster}{}%
Eklund, A., Nichols, T. E., \& Knutsson, H. (2016). Cluster failure: Why
fMRI inferences for spatial extent have inflated false-positive rates.
\emph{Proceedings of the National Academy of Sciences}, \emph{113}(28),
7900--7905.

\leavevmode\hypertarget{ref-gates2011extended}{}%
Gates, K. M., Molenaar, P. C., Hillary, F. G., \& Slobounov, S. (2011).
Extended unified sem approach for modeling event-related fMRI data.
\emph{NeuroImage}, \emph{54}(2), 1151--1158.

\leavevmode\hypertarget{ref-hershberger2003growth}{}%
Hershberger, S. L. (2003). The growth of structural equation modeling:
1994-2001. \emph{Structural Equation Modeling}, \emph{10}(1), 35--46.

\leavevmode\hypertarget{ref-joreskog1996lisrel}{}%
Joreskog, K., \& Sorbom, D. (1996). LISREL 8: User's reference guide
(scientific software international, chicago). \emph{Google Scholar}.

\leavevmode\hypertarget{ref-kim2007unified}{}%
Kim, J., Zhu, W., Chang, L., Bentler, P. M., \& Ernst, T. (2007).
Unified structural equation modeling approach for the analysis of
multisubject, multivariate functional mri data. \emph{Human Brain
Mapping}, \emph{28}(2), 85--93.

\leavevmode\hypertarget{ref-muthen2010mplus}{}%
Muthén, L. K., \& Muthén, B. O. (2010). \emph{Mplus: Statistical
analysis with latent variables: User's guide}. Muthén \& Muthén Los
Angeles.

\leavevmode\hypertarget{ref-rosseel2012lavaan}{}%
Rosseel, Y. (2012). Lavaan: An r package for structural equation
modeling and more. Version 0.5--12 (beta). \emph{Journal of Statistical
Software}, \emph{48}(2), 1--36.






\end{document}
