\documentclass[jou]{apa6}

\usepackage{amssymb,amsmath}
\usepackage{ifxetex,ifluatex}
\usepackage{fixltx2e} % provides \textsubscript
\ifnum 0\ifxetex 1\fi\ifluatex 1\fi=0 % if pdftex
  \usepackage[T1]{fontenc}
  \usepackage[utf8]{inputenc}
\else % if luatex or xelatex
  \ifxetex
    \usepackage{mathspec}
    \usepackage{xltxtra,xunicode}
  \else
    \usepackage{fontspec}
  \fi
  \defaultfontfeatures{Mapping=tex-text,Scale=MatchLowercase}
  \newcommand{\euro}{€}
\fi
% use upquote if available, for straight quotes in verbatim environments
\IfFileExists{upquote.sty}{\usepackage{upquote}}{}
% use microtype if available
\IfFileExists{microtype.sty}{\usepackage{microtype}}{}

% Table formatting
\usepackage{longtable, booktabs}
\usepackage{lscape}
% \usepackage[counterclockwise]{rotating}   % Landscape page setup for large tables
\usepackage{multirow}		% Table styling
\usepackage{tabularx}		% Control Column width
\usepackage[flushleft]{threeparttable}	% Allows for three part tables with a specified notes section
\usepackage{threeparttablex}            % Lets threeparttable work with longtable

% Create new environments so endfloat can handle them
% \newenvironment{ltable}
%   {\begin{landscape}\begin{center}\begin{threeparttable}}
%   {\end{threeparttable}\end{center}\end{landscape}}

\newenvironment{lltable}
  {\begin{landscape}\begin{center}\begin{ThreePartTable}}
  {\end{ThreePartTable}\end{center}\end{landscape}}

  \usepackage{ifthen} % Only add declarations when endfloat package is loaded
  \ifthenelse{\equal{\string jou}{\string man}}{%
   \DeclareDelayedFloatFlavor{ThreePartTable}{table} % Make endfloat play with longtable
   % \DeclareDelayedFloatFlavor{ltable}{table} % Make endfloat play with lscape
   \DeclareDelayedFloatFlavor{lltable}{table} % Make endfloat play with lscape & longtable
  }{}%



% The following enables adjusting longtable caption width to table width
% Solution found at http://golatex.de/longtable-mit-caption-so-breit-wie-die-tabelle-t15767.html
\makeatletter
\newcommand\LastLTentrywidth{1em}
\newlength\longtablewidth
\setlength{\longtablewidth}{1in}
\newcommand\getlongtablewidth{%
 \begingroup
  \ifcsname LT@\roman{LT@tables}\endcsname
  \global\longtablewidth=0pt
  \renewcommand\LT@entry[2]{\global\advance\longtablewidth by ##2\relax\gdef\LastLTentrywidth{##2}}%
  \@nameuse{LT@\roman{LT@tables}}%
  \fi
\endgroup}


\ifxetex
  \usepackage[setpagesize=false, % page size defined by xetex
              unicode=false, % unicode breaks when used with xetex
              xetex]{hyperref}
\else
  \usepackage[unicode=true]{hyperref}
\fi
\hypersetup{breaklinks=true,
            pdfauthor={},
            pdftitle={A Commentary on Mueller (1997)},
            colorlinks=true,
            citecolor=blue,
            urlcolor=blue,
            linkcolor=black,
            pdfborder={0 0 0}}
\urlstyle{same}  % don't use monospace font for urls

\setlength{\parindent}{0pt}
%\setlength{\parskip}{0pt plus 0pt minus 0pt}

\setlength{\emergencystretch}{3em}  % prevent overfull lines


% Manuscript styling
\captionsetup{font=singlespacing,justification=justified}
\usepackage{csquotes}
\usepackage{upgreek}



\usepackage{tikz} % Variable definition to generate author note

% fix for \tightlist problem in pandoc 1.14
\providecommand{\tightlist}{%
  \setlength{\itemsep}{0pt}\setlength{\parskip}{0pt}}

% Essential manuscript parts
  \title{A Commentary on Mueller (1997)}

  \shorttitle{A Commentary on Mueller (1997)}


  \author{Chi-Lin Yu\textsuperscript{1}}

  % \def\affdep{{""}}%
  % \def\affcity{{""}}%

  \affiliation{
    \vspace{0.5cm}
          \textsuperscript{1} Department of Psychology, National Taiwan University  }

  \authornote{
    The present commentary is a homework of Structural Equation Modeling
    course in Department of Psychology, National Taiwan University. Mentor:
    Li-Jen Weng. Student ID: R05227101 (Chi-Lin Yu). E-mail address:
    \href{mailto:r05227101@ntu.edu.tw}{\nolinkurl{r05227101@ntu.edu.tw}};
    \href{mailto:psychilinyu@gmail.com}{\nolinkurl{psychilinyu@gmail.com}}.
    Draft can be found on \url{https://github.com/PsyChiLin}.
  }


  




\usepackage{amsthm}
\newtheorem{theorem}{Theorem}
\newtheorem{lemma}{Lemma}
\theoremstyle{definition}
\newtheorem{definition}{Definition}
\newtheorem{corollary}{Corollary}
\newtheorem{proposition}{Proposition}
\theoremstyle{definition}
\newtheorem{example}{Example}
\theoremstyle{definition}
\newtheorem{exercise}{Exercise}
\theoremstyle{remark}
\newtheorem*{remark}{Remark}
\newtheorem*{solution}{Solution}
\begin{document}

\maketitle

\setcounter{secnumdepth}{0}



\setlength{\parindent}{4ex}

Structural equation modeling (SEM) is a popular statistical procedure in
social science. Lots of technological advances have provided users a
variety of tools to perform SEM, even though users may not really
familiar with the underlying statistical theory and building blocks of
SEM (Mueller, 1997). Mueller (1997) provide a series of reminders on
some of neglected topics, including specification, identification,
estimation, and evaluation. Also, Mueller (1997) mentioned the
software-driven properties of SEM. The present article aims to provide a
serious discussion regarding the software-driven properties of SEM.

Mueller (1997) reviewed the progress of major software packages, such as
LISREL (Joreskog \& Sorbom, 1996) and EQS (Bentler, 1995). Previously,
users have to completely understand several complex matrix in order to
obtain a result. Nowadays, users only have to click some slide bars or
some pull down menus through user-friendly graphical user interface
(GUI). However, these user-friendly tools may not be always good because
users without any understanding of SEM can also easily use the tool and
propose some results. Some mistakes may be made and may hamper the
development of scientific theory. I really believe that this phenomenon
exists not only in SEM but in every domains. For example, in functional
Magnetic Resonance Imaging (fMRI) domain, the most popular package is
SPM, Statistical Parametric Mapping. Lots of researcher used SPM to
analyze their collected fMRI data. However, SPM is a black box. Only
imaging experts with MATLAB programming skills know how to build up and
how SPM works. I am confident that at least half of people used SPM do
not understand how it works, but we still use it to analyze data and
interpret results. Does it hamper the development of neuroimaging
research ? Not really. It actually makes the fMRI data more attractive
and more accessible. Does it produce problems ? Yes, it does.
Methodological issues always exists in fMRI data due to the problematic
usage of SPM. Another example is SPSS, which is the most popular
statistical software in the world. Tons of psychologist used SPSS,
although it is also a black box with GUI. Anyone without clear
understandings of a statistical method can use SPSS to produce some
results. There is a meme created to discourage the usage of SPSS (Figure
1). The same situation can also be suitably applied to the development
of SEM. An user-friendly software is a double-edged sword. It can
provide a successful progress in SEM, but also let users forget some
important topics of SEM.

\begin{figure}

{\centering \includegraphics[width=3.03in]{SPSS} 

}

\caption{Memes of SPSS}\label{fig:SPSS}
\end{figure}

In my opinion, I think that there are some potential solutions to
partially address these issues. First, some exploratory procedures
should be provided. Without going through the exploratory data analysis,
the software should not allow users to go to the next step. Second, some
automatic detections, e.g., distribution, co-linearity, identifications,
can be provided. It may provide an opportunity for users to reconsider
heir dataset and their model. For example, if a model users specified is
not identified, the software should reject the model and require
adjustments from users. Third, based on some automatic detections and
exploratory data analysis, some appropriate defaults can be provided.
For example, if the input data has the properties of multivariate
normality and has enough sample size, maximum likelihood estimation can
be automatically set as a default. If not, other corrected method can be
automatically adjusted as a default. Finally, these procedure should be
integrated into a single platform. Users could easily input their data
and get the results with minimum errors and efforts. Whether or not this
kind of software will obstruct the way of a researcher to become a SEM
expert is not clear (Users who mistakenly use other softwares may not be
interested in being a SEM expert. They just want to apply.), but what is
clear is that it will let the produced results with minimum errors. I
believe that this kind of software developments may have its importance
in advancing the scientific progress. In fact, previous research have
tried this direction on exploratory factor analysis (EFA) and provided a
user-friendly applications with good default arguments (Yu \& Sheu,
2018). Maybe it can somehow be applied to SEM community.

However, any adjustments or improvements of software can only address
the easy part of SEM. \enquote{The hard part is constructing causal
models that are consistent with sound theory} (Mueller, 1997). It is not
only the most hard part but also the most important part, because SEM
can only benefit the model with strong theoretical background. Also,
this hardest part can not be trained or addressed within a few days. It
required a long time to establish the well understandings toward a
domain. Optimistically speaking, although this part is really hard and
important, I think that it will not be an issue for behavioral scientist
or psychologist who have solid domain knowledge. A researcher with solid
background, combined with an useful software or a SEM expert, would be
able to deal with any statistical and theoretical issue.

In summary, SEM, like many other methodologies, is software-driven. This
phenomenon will not change and will keep progress as long as the
increasing of SEN techniques. The present commentary provide some own
perspectives to somehow partially address the gap between users and this
method. In the near future, maybe some advanced softwares with clear
guides can actually be developed for analyses purposes and tutorial
purposes. Definitely, do not forget to back to theortical formulations,
the basis and most important part of SEM.

\hypertarget{references}{%
\section{References}\label{references}}

\setlength{\parindent}{-0.5in}
\setlength{\leftskip}{0.5in}

\hypertarget{refs}{}
\leavevmode\hypertarget{ref-bentler1995eqs}{}%
Bentler, P. M. (1995). \emph{EQS structural equations program manual}.
Multivariate software.

\leavevmode\hypertarget{ref-joreskog1996lisrel}{}%
Joreskog, K., \& Sorbom, D. (1996). LISREL 8: User's reference guide
(scientific software international, chicago). \emph{Google Scholar}.

\leavevmode\hypertarget{ref-mueller1997structural}{}%
Mueller, R. O. (1997). Structural equation modeling: Back to basics.
\emph{Structural Equation Modeling: A Multidisciplinary Journal},
\emph{4}(4), 353--369.

\leavevmode\hypertarget{ref-yuefashiny}{}%
Yu, C.-L., \& Sheu, C.-F. (2018). EFAshiny: An user-friendly shiny
application for exploratory factor analysis. \emph{The Journal of Open
Source Software}, \emph{3}(22), 567.






\end{document}
